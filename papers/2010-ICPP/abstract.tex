\begin{abstract}
\label{section:abstract}

The size of datasets produced by current climate models is increasing rapidly
to the scale of petabytes.  To handle data at this scale parallel analysis
tools are required, however the majority of climate analysis software remains
at the scale of workstations.  Further, many climate analysis tools adequately
process regularly gridded data but lack sufficient features when handling
unstructured grids.  This paper presents a data-parallel subsetter capable of
correctly handling unstructured grids while scaling to over 2000 cores.  The
approach is based on the partitioned global address space (PGAS) parallel
programming model and one-sided communication.  The paper demonstrates that IO
remains the single greatest bottleneck for this domain of applications and
that parallel analysis of climate data succeeds in practice.

\end{abstract}
