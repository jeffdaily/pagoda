\section{Future Work}
\label{section:future}

The success of the NetCDF Operators\cite{NCO} and similar tools demonstrate
the need for user-ready applications for the analysis of their users' data.  
The success of tools such as CDAT\cite{CDAT} validate the need for a
scriptable interface and customization of basic and advanced operators.  We
plan to provide both the scriptable interface as well as a set of predefined
command-line tools.

We are currently developing a general C++ API for climate data analysis in a
data parallel fashion based on the PGAS model and one-sided communications.
The API will be leveraged to produce additional command-line tools, however it
is intended primarily to be used by climate scientists to produce the kinds of
tailored analyses which they require.  Time permitting, the API will be
exposed to the Python language in order to facilitate ease of use in a
scripted environment.  Larson, Ong, and Tokarz also point out certain
capabilities missing from popular climate data analysis packages such as
probability density function (PDF) estimation as well as the sorting and
ordering of data.

The evaluation performed in Section \ref{section:evaluation} revealed that IO
for our application remains the greatest bottleneck.  This fact is exacerbated
when small regions representing a mere fraction of the entire grid are subset.
Our current strategy is to read in entire variables and in the case of small
subsets immediately cull the majority of them.  If a masked read or a read
based on individual array index tuples similar to \verb+NGA_Gather+ were
available within the Parallel-NetCDF library we would likely see performance
gains.  The algorithms presented in this paper which evenly distribute the
subset data will hopefully encourage work in this area.  We are currently
evaluating a strategy of non-collective reads for only those processes which
would eventually participate in the subset.
