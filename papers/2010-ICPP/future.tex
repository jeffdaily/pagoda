\section{Future Work}
\label{section:future}

The success of the NetCDF Operators\cite{NCO} and similar tools demonstrate
the need for user-ready applications for the analysis of their users' data.
The success of tools such as CDAT\cite{CDAT} validate the need for a
scriptable interface and customization of basic and advanced operators.  We
plan to provide both the scriptable interface as well as a set of predefined
command-line tools.

We are currently developing a general C++ API for climate data analysis in a
data parallel fashion based on the PGAS model and one-sided communication.
The API will be leveraged to produce additional command-line tools, however it
is intended primarily to be used by climate scientists to produce the kinds of
tailored analyses which they require.  Time permitting, the API will be
exposed to the Python language in order to facilitate ease of use in a
scripted environment.  Larson, Ong, and Tokarz also point out certain
capabilities missing from popular climate data analysis packages such as
probability density function (PDF) estimation as well as the sorting and
ordering of data.

The evaluation performed in Section \ref{section:evaluation} revealed that IO
for our application remains the greatest bottleneck.  This fact is exacerbated
when large regions representing the entire grid are subset.  Our current
optimized strategy of reading chunks of variables based on whether a process
participates in the later subset still reads more data than is eventually
redistributed.  If a masked read or a read based on individual array index
tuples similar to \verb+NGA_Gather+ were available within the Parallel-NetCDF
library we might see further performance gains.  Work along these lines has
been suggested as part of the future development of Parallel-NetCDF
\cite{PNETCDFOPT}.
