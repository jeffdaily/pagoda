\section{Introduction}
\label{section:introduction}

Parallel programming is needed to analyze the size of output data produced by
today's climate models\cite{MODSIM07:LOT,WASHINGTON08}.  A single snapshot of
a Global Cloud Resolving Model (GCRM) at 4 kilometer resolution will produce
terabytes of data\cite{GCRM}; the analysis of even a modest time series of
this data will quickly overwhelm today's software and traditional climate
analysis systems.  For these data sizes, IO bandwidth represents the single
greatest bottleneck for analysis tools.  Parallel software leveraging parallel
file systems must be used to process this data, however current climate
analysis tools are at most task parallel and rely on a single data
reader\cite{CDAT,CDO,NCO}.

Many climate analysis tools robustly handle the manipulation and display of
regularly gridded data.  However, these same applications lack sufficient
features when handling unstructured or irregular grids such as the
geodesic\cite{GEODESIC} or cubed sphere\cite{CUBE} grids.  Unstructured grids
are gaining popularity, further widening the gap between current software and
these types of models.  For unstructured grids it is necessary to provide more
information about the topology of the grid and to maintain the integrity of this
topology information in the face of data culling.

Regular grids allow for the topology to be implicitly defined by how
the data is stored; coordinate variables are generally monotonic and cell
neighbors are adjacent both logically and in memory.  These assumptions allow
for operations over regular grids which are otherwise more difficult to
perform over unstructured grids.  In the case of partitioning these grids for
data parallel processing, unstructured grids will often have more of the
logically adjacent cells scattered across memory partitions than in the
regular case.

Subsetting is a fundamental capability for any analysis tool and allows users
to operate over regions of the data in which they are interested.  The
subsetting operation is useful as part of a larger operation over the data,
such as obtaining regional averages, but is also useful to post-process data
into a new dataset such that the cost of subsetting can be amortized across
future operations over the same region.  Further, as the size of datasets
grow, subsetting is important to reduce the data to a size that traditional
analysis tools are capable of handling. 

In this paper, we present a parallel tool for subsetting very large climate
data generated on a geodesic grid while preserving the explicit topology.  The
code is built using the Global Arrays (GA) toolkit\cite{GA} which provides an
efficient and portable "shared-memory" programming interface for
distributed-memory computers and features truly one-sided communications.  GA
traditionally represents dense arrays, however its sparse data operations over
one-dimensional arrays as well as its one-sided operations allow for efficient
subsetting over unstructured grids.

The primary contributions of the paper are:
\begin{itemize}
\item A parallel subsetter of geodesic data based on the partitioned global
address space (PGAS) programming model and one-sided communications
\item Novel algorithms for the maintenance of unstructured grid topology data
\item A novel algorithm for the subset and even distribution of unstructured grid data
\item Evaluation showing IO to be the greatest bottleneck in scaling these types of applications
\end{itemize}

The paper is organized as follows.  Section \ref{section:requirements}
describes the requirements while Section \ref{section:design} describes the
design of the subsetter, its algorithms, and how GA's unique features were
leveraged.  Section \ref{section:evaluation} presents our experimental design
and the performance characteristics of the subsetter on nearly full-scale
dataset sizes of model data up to a resolution of 4 kilometer.  We present the
capabilities under development as well as the capabilities we would like to
see in Section \ref{section:future}.  Finally, Section
\ref{section:conclusion} presents our conclusions.
